% Inoffizielle LaTeX-Vorlage für Vorträge
% an der Fakultät für Chemie und Pharmazie
% der Albert-Ludwigs-Universität Freiburg
%
% Diese Vorlage ist lediglich ein Vorschlag, der versucht, sowohl
% typografischen Ansprüchen ansatzweise gerecht zu werden als auch
% nutzbar zu sein.
%
% Jegliche Nutzung auf eigene Verantwortung.
%
% Copyright (c) 2018, Till Biskup

% Schriftart fuer Formeln anpassen: Serifenschrift
\usefonttheme[onlymath]{serif}

% Vektoren: fett, kursiv
\usepackage{bm}
\renewcommand*{\vec}[1]{\bm{#1}}

% Tensoren: serifenlos, fett, kursiv
\DeclareMathAlphabet{\mathbfsf}{\encodingdefault}{\sfdefault}{bx}{sl}
\newcommand*{\tens}[1]{\mathbfsf{#1}}

% Differentialoperator: aufrecht
\newcommand*{\diff}{\mathrm{d}}

% mathematische Konstanten: aufrecht
\newcommand*{\im}{\mathrm{i}}
\newcommand*{\e}{\mathrm{e}}

% für das aufrechte kleine pi: \uppi
\usepackage[Symbolsmallscale]{upgreek}

% Operatoren mit Dach
\newcommand*{\op}[1]{\hat{#1}}

% Dirac-Notation
\newcommand*{\bra}[1]{\left\langle#1\right\rvert}
\newcommand*{\ket}[1]{\left\lvert#1\right\rangle}
\newcommand*{\braket}[2]{\langle#1\lvert#2\rangle}
